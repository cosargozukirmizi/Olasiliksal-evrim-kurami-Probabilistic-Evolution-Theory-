\documentclass[a4paper,10pt]{article}
\renewcommand{\refname}{Kaynaklar}
\usepackage[utf8]{inputenc}
\usepackage[a4paper, total={6.5in, 10.0in}]{geometry}
\usepackage[FIGTOPCAP]{subfigure}
\usepackage{tikz-qtree}
\usetikzlibrary{graphdrawing, graphs}
\usepackage{tikz}
\usepackage{framed}
\usegdlibrary{trees, layered}
\usepackage{hyperref}
\usepackage{doi}

%opening
\title{D\"ord\"ulle\c{s}tirilmi\c{s} \i rakg\"or\"ur dizeylerinin 
belirlenimi}
\author{Co\c{s}ar G\"oz\"uk\i rm\i z{\i}}
\date{30 Aral\i k 2015}
\linespread{1.5}
\renewcommand{\figurename}{\c{C}izim}
\begin{document}
\maketitle

\section{Giri\c{s}}

Olas\i l\i ksal evrim kuram{\i} (OEK) ba\u{g}lam\i nda yap\i lan %
\c{c}al\i \c{s}malar \"ozellikle sa\u{g} yan{\i} ikinci de\-rece %
\c{c}ok\c{c}okterimli olan belirtik s\i radan t\"urevli denklem t%
ak\i mlar\i n\i n ba\c{s}lang\i \c{c} de\u{g}er so\-rununun \c{c}%
\"oz\"um\"u i\c{c}in \"onemli ayg\i tlar olu\c{s}turumuna olanak %
sa\u{g}lamaktad\i r. Birim dizeyler ile Kronecker \c{c}arp\i m ku%
llanarak dizey boyutlar\i n\i n b\"uy\"ut\"uld\"u\u{g}\"u yap\i d%
an ka\c{c}\i nmak i\c{c}in d\"ord\"ulle\c{s}tirim olgusu g\"undem%
e getirilmi\c{s}tir. D\"ord\"ulle\c{s}tirilmi\c{s} \i rakg\"or\"u%
r dizeylerinin kullan\i m{\i}, boyutlar{\i} de\u{g}i\c{s}meyen di%
zey ve y\"oneylerle \c{c}al\i \c{s}may{\i} olanakl{\i} k\i lmakta%
d\i r.

\section{Olas\i l\i ksal evrim kuram{\i}}

Olas\i l\i ksal evrim kuram{\i} ile sa\u{g} yan{\i} ikinci de\-re%
ce \c{c}ok\c{c}okterimli olan belirtik std \c{c}\"oz\"um\"u %
%------------------------------(1)-----------------------------------%
\begin{equation}
\mathbf{x} = {\rm e}^{\beta t}
\sum\limits_{j=0}^{\infty}\frac{1}{j!}%
\left(\frac{{\rm e}^{\beta t}-1}{\beta}\right)^{j}%
\mathbf{T}_{j}
\mathbf{a}^{\otimes (j+1)}
\label{eq:XNCbh}
\end{equation}
%------------------------------(1)-----------------------------------%
bi\c{c}imindedir. D\"ord\"ulle\c{s}tirim kullan{\i}m{\i} ise %
%------------------------------(2)-----------------------------------%
\begin{equation}
 \mathbf{T}_{j} \mathbf{a}^{\otimes (j+1)} = 
 \mathbf{S}_{j}(\mathbf{a})
 \mathbf{a}, \quad j=0,1,\ldots
\label{eq:AN8En}
\end{equation}
%------------------------------(2)-----------------------------------%
yap\i s\i n{\i} \"ong\"or\"ur. D\"or\-d\"ul\-le\c{s}ti\-rimin tan%
\i m{\i} %
%------------------------------(3)-----------------------------------%
\begin{equation}
 \lfloor \mathbf{F} , \mathbf{a} \rceil = 
 \sum_{i=1}^{n} a_{i} \mathbf{F}^{(i)}
\label{eq:vWTnF}
\end{equation}
%------------------------------(3)-----------------------------------%
bi\c{c}imindedir. $\mathbf{F}^{(i)}$ ile belirtilen d\"ord\"ul di%
zeyler, $\mathbf{F}$ dizeyinin \"obekleridir. Ge\-rek\-li olan ol%
gu $\mathbf{S}_{j}(\mathbf{a})$ dizeylerinin d\"ord\"ulle\c{s}tir%
imler bi\c{c}iminden yaz\i m\i d\i r. %

\section{D\"ord\"ulle\c{s}tirim}

OEK ba\u{g}lam\i nda g\"undeme gelen \i rakg\"or\"ur dizeyleri da%
ha belirtik bir anlat\i mla g\"undeme getirilebilir. %

Irakg\"or\"ur dizeylerinin yap\i s{\i}, %
%------------------------------(4)-----------------------------------%
\begin{equation}
{\bf T}_j=\prod_{k=1}^j{\bf M}_k
\end{equation}
%------------------------------(4)-----------------------------------%
bi\c{c}imindedir. Irakg\"or\"ur\"un b\"ol\"umleri ise %
%------------------------------(5)-----------------------------------%
\begin{equation}
{\bf M}_j=\sum_{k=0}^{j-1}\left({\bf I}_n^{\,\otimes k}%
\otimes{\bf F}\otimes{\bf I}_n^{\,\otimes j-k-1}\right) 
\end{equation}
%------------------------------(5)-----------------------------------%
olarak g\"undeme gelir. %

En ba\c{s}taki birka\c{c} terim %
%------------------------------(6)-----------------------------------%
\begin{equation}
{\bf M}_1 = {\bf F}
\end{equation}
%------------------------------(6)-----------------------------------%
%------------------------------(7)-----------------------------------%
\begin{equation}
{\bf M}_2 = \left({\bf F}\otimes{\bf I}\right)
+\left({\bf I}\otimes{\bf F}\right)
\end{equation}
%------------------------------(7)-----------------------------------%
%------------------------------(8)-----------------------------------%
\begin{equation}
{\bf M}_3 = \left({\bf F}\otimes{\bf I}^{\otimes 2}\right)
+\left({\bf I}\otimes{\bf F}\otimes{\bf I}\right)
+\left({\bf I}^{\otimes 2}\otimes{\bf F}\right)
\end{equation}
%------------------------------(8)-----------------------------------%
bi\c{c}imindedir. Tan\i m olarak %
%------------------------------(9)-----------------------------------%
\begin{equation}
 {\bf T}_{0} =  {\bf I}
\end{equation}
%------------------------------(9)-----------------------------------%
olarak g\"undeme gelmektedir. Yi\-ne ko\-layl\i kla g\"or\"ulebil%
ece\u{g}i gibi %
%------------------------------(10)-----------------------------------%
\begin{equation}
 {\bf T}_{1} = {\bf F}
\end{equation}
%------------------------------(10)-----------------------------------%
yap\i s\i ndad\i r. Bir sonraki \i rakg\"or\"ur dizeyi ise %
%------------------------------(11)-----------------------------------%
\begin{eqnarray}
 {\bf T}_{2} &=& {\bf F}\big[\left({\bf F}\otimes{\bf I}\right)
+\left({\bf I}\otimes{\bf F}\right)\big] \nonumber\\
&=&{\bf F}\left({\bf F}\otimes{\bf I}\right)+
{\bf F}\left({\bf I}\otimes{\bf F}\right)
\label{eq:aqweq}
\end{eqnarray}
%------------------------------(11)-----------------------------------%
olarak kar\c{s}\i m\i za \c{c}\i kar. Benzer inceleyi\c{s} ile, %
%------------------------------(12)-----------------------------------%
\begin{eqnarray}
{\bf T}_{3} &=& {\bf F} \big[\left({\bf F}\otimes{\bf I}\right)
+\left({\bf I}\otimes{\bf F}\right)\big]
\big[ \left({\bf F}\otimes{\bf I}^{\otimes 2}\right)
+\left({\bf I}\otimes{\bf F}\otimes{\bf I}\right)
+\left({\bf I}^{\otimes 2}\otimes{\bf F}\right) \big]
\end{eqnarray}
%------------------------------(12)-----------------------------------%
elde edilebilir.

$\mathbf{S}_{j}(\mathbf{a})$ dizeylerinin bulunumu i\c{c}in ilgil%
i yap\i lar{\i} kul\-la\-narak ve \c{c}arp\i m ve Kronecker \c{c}%
arp\i m\i n birbirleri \"uzerinde da\u{g}\i lma \"ozelli\u{g}ini %
g\"oz \"on\"unde bulundurarak iler\-le\-ne\-bilir. Bu bi\-\c{c}im%
de, $\mathbf{S}_{j}(\mathbf{a})$ dizeyleri e\c{s}siz bi\c{c}imde %
elde edilebilir. Bu bulunum, cebircil in\-dirgeyi\c{s} i\-\c{c}er%
ir. Cebircil indirdeyi\c{s}in sa\-y\i c\i l o\-larak bi\c{c}eleni%
\c{s}i, bulunumu kolayla\c{s}t\i r\i r. D\"ord\"ulle\c{s}tirim ce%
brinin say\i c\i l bi\c{c}e\-le\-ni\-\c{s}i, \c{c}a\-l\i \c{s}man%
\i n \c{c}ekirde\u{g}ini olu\c{s}turmaktad\i r.

(\ref{eq:aqweq}) ba\u{g}\i nt\i s\i nda $\mathbf{T}_{2}$ i\c{c}in a%
\c{c}\i k\c{c}a g\"osterildi\u{g}i gibi, b\"ut\"un \i rakg\"or\"u%
r dizeylerinin toplamc\i l terimleri vard\i r. Her toplamc\i l te%
rim, $\mathbf{F}$'nin i\c{c}inde bir kere bulundu\u{g}u \c{c}arpa%
nlardan olu\c{s}ur. Bu \c{c}arpanlar\-da i\-se $\mathbf{F}$'nin s%
a\u{g}\i nda ve solunda $0$ ya da daha b\"uy\"uk say\i da birim d%
izeyin Kronecker \c{c}arp\i m{\i} bulunmaktad\i r. Bi\-rim dizeyi%
n Kronecker \"usl\"us\"u yerine, bu birim dizeylerin ardarda Kron%
ecker \c{c}arp\i m ile belirtik olarak ya\-z\i ld\i \u{g}{\i} yap%
{\i} d\"u\c{s}\"un\"ul\"urse, her bir \c{c}arpanda $\mathbf{F}$'n%
in konumu bir tamsay{\i} ile g\"osterilebilir. Bu i\c{s}lem, her %
bir terim i\c{c}in yinelenebilir. $\mathbf{T}_{2}$ i\c{c}in somut%
la\c{s}t\i rmak gerekirse, $\mathbf{T}_{2}$ iki di\-zi\-lim ile e%
\c{s}siz bi\c{c}imde nitelendirilebilir. Bu dizilimler %
%------------------------------(13)-----------------------------------%
\begin{eqnarray}
 1\quad 2 \nonumber\\
 1\quad 1
\end{eqnarray}
%------------------------------(13)-----------------------------------%
dizilimleridir. Dizilimlerde birince say{\i}, birinci \c{c}arpand%
aki $\mathbf{F}$'nin yerini, ikinci say{\i} ise ikinci \c{c}arpan%
daki $\mathbf{F}$'nin yerini g\"osterir. Asl\i nda birinci \c{c}a%
rpanda yaln\i zca bir yer oldu\u{g}u i\c{c}in, dizilimin birinci %
\"o\u{g}esi yaln\i zca $1$ de\u{g}erini alabilir. Dizilimin ikinc%
i \"o\u{g}esi ise $1$ ve $2$ de\u{g}er\-lerini alabilir. Burada g%
\"ozlemlenmesi gereken olgu, di\-zilimlerin bir alt\"u\c{c}\-gen %
i\c{c}erisinde g\"undeme gelebilecek b\"ut\"un ya\-p\i lar ol\-du%
\u{g}udur. %

Benzer bir inceleme ile, $\mathbf{T}_{3}$ i\c{c}in %
%------------------------------(14)-----------------------------------%
\begin{eqnarray}
 1\quad 2 \quad 3 \nonumber\\
 1\quad 2 \quad 2 \nonumber\\
 1\quad 2 \quad 1 \nonumber\label{eq:121}\\
 1\quad 1 \quad 3 \nonumber\label{eq:113}\\
 1\quad 1 \quad 2 \nonumber\\
 1\quad 1 \quad 1
\end{eqnarray}
%------------------------------(14)-----------------------------------%
elde edilebilir. $\mathbf{T}_{2}$'deki b\"ut\"un dizilimlerin aza%
lmayan nitelikte ol\-ma\-s\i na kar\c{s}\i n, $\mathbf{T}_{3}$'de%
ki $1\; 2\; 1$ dizilimi azalan nitelik\-tedir. $\mathbf{T}_{3}$ i%
\c{c}in, b\"ut\"un dizilimlere kar\c{s}\i l\i k gelen \c{c}arpanl%
ar ilgili y\"oney Kronecker \"usl\"us\"u i\-le ay\-r{\i} ayr{\i} %
\c{c}arp\i l\i r ise, $1\; 2\; 1$ ve $1\; 1\; 3$ dizilimlerine ka%
r\c{s}\i l\i k gelen yap\i lar\i n ayn{\i} katk\i y{\i} \"urettik%
leri g\"ozlemlenebilir. Bu olgu d\"ord\"ulle\c{s}tirim cebrinde k%
endini a\c{c}\i k olarak g\"ostermektedir. Bunun nedeni, \c{c}arp%
anlar aras\i nda \"ozel bir de\u{g}i\c{s}tirimin ge\c{c}erli olma%
s\i d\i r. Bu bulgu somut olarak %
%------------------------------(15)-----------------------------------%
\begin{equation}
 {\bf F}\left({\bf I}\otimes{\bf F}\right)%
 \left({\bf F}\otimes{\bf I}^{\otimes 2}\right)
 {\bf a}^{\otimes 4} = {\bf F}\left({\bf F}\otimes{\bf I}\right)
 \left({\bf I}^{\otimes 2}\otimes{\bf F}\right){\bf a}%
^{\otimes 4}
\end{equation}
%------------------------------(15)-----------------------------------%
olarak g\"osterilebilir. Bu iki katk{\i} birbirine e\-\c{s}it old%
u\u{g}u i\c{c}in, bir kez belirleyip, $2$ kat\-sa\-y{\i}\-s{\i} i%
\-le \c{c}arpmak daha verimli bir yol olarak g\"or\"ulmektedir. %

Ayn{\i} katk\i y{\i} \"uretecek olan dizilimler, dizilimlerin ken%
dilerini kullanarak somut ad\i mlarla belirlenebilir ve bu ad\i m%
lar asl\i nda d\"ord\"ulle\c{s}tirim ceb\-ri\-nin bir yans{\i}mas%
{\i}d\i r. Terimlerin tamsay{\i} katsay\i lar\i n\i n bulunumu ol%
arak da de\u{g}erlendirilebilecek bu i\c{s}lem i\c{c}in \"onerile%
n y\"ontem kabarc\i kl{\i} s\i ralay{\i}\c{s}{\i} temel alan \"oz%
el kabarc\i kl{\i} s\i ralay\i \c{s} y\"ontemidir. Ka\-barc{\i}kl%
{\i} s\i ralay\i \c{s}, dizilimdeki arda\c{s}\i k \"o\u{g}elerin %
kar\c{s}\i la\c{s}t\i r\i l\i p, e\u{g}er s\i raya ay\-k{\i}r{\i}%
l{\i}k s\"oz konusu ise, \"o\u{g}elerin yer de\u{g}i\c{s}\-tirimi%
ne dayan\i r. Bu i\c{s}lem azalmayan s\i ralan\i m elde edene kad%
ar yinelenir. \"Ozel kabarc\i kl{\i} s\i ralay\i \c{s} i\-le kaba%
rc\i kl{\i} s\i ralay\i \c{s} aras\i nda \"onemli ve yal\i n bir %
ayr\i m vard\i r. \"Ozel kabarc\i kl{\i} s\i ralay\i \c{s}ta yer %
de\u{g}i\c{s}tirimi yap\i l\i rken, so\-la aktar\i lan \"o\u{g}en%
in de\u{g}erinin korunmas\i na kar\c{s}\i n, sa\u{g}a aktar{\i}la%
n \"o\u{g}enin de\u{g}eri bir art\i r\i l\i r. Dolay{\i}s\i yla, %
\"ozel kabarc\i kl{\i} s\i ralay\i \c{s} \"o\u{g}e de\u{g}erlerin%
i de\u{g}i\c{s}tirir. %

$\mathbf{T}_{3}$ ba\u{g}lam{\i}nda g\"un\-de\-me ge\-len diziliml%
er i\c{c}in bu olguyu \"orneklendirelim. Yap\i lmas{\i} gereken s%
\i ral{\i} olmayan dizilimlerin belirlenip, bu di\-zilimlerin han%
g{\i} s\i ral{\i} (azalmayan) dizilimlerle e\c{s}de\u{g}erli oldu%
\u{g}unun bulunmas{\i}d\i r. $\mathbf{T}_{3}$ i\c{c}in yal\-n{\i}%
zca bir adet azalabilen dizilim bulunmaktad\i r: $1\; 2\; 1$. \"O%
zel kabarc\i kl{\i} s\i ralay\i \c{s}\i n ad\i mlar{\i} \c{s}u bi%
\c{c}imdedir. %
\begin{itemize}
 \item 1 ile 2 s\i ral{\i} m{\i}: Evet.
 \item 2 ile 1 s\i ral{\i} m{\i}: Hay\i r. $2$'ye $1$ ekle ve sa%
 \u{g}\i ndaki $1$ ile yer de\u{g}i\c{s}tir.
 Yeni dizilim: $1\; 1\; 3$.
 \item 1 ile 1 s\i ral{\i} m{\i}: Evet.
 \item 1 ile 3 s\i ral{\i} m{\i}: Evet.
\end{itemize}
Bu ba\u{g}lamda $1\; 2\; 1$'in \"ozel ka\-bar\-c\i kl{\i} s\i ral%
ay\i \c{s}tan ge\c{c}mi\c{s} bi\c{c}imi $1\; 1\; 3$ di\-zi\-limid%
ir. Dolay\i s\i yla, alt\"u\c{c}gendeki azalmayan dizilimleri olu%
\c{s}turup, bu dizilimlerin herbirine bir de tam\-sa\-y{\i} katsa%
y{\i} e\c{s}le\c{s}tirmek yoluna gidi\-lebilir. $\mathbf{T}_{3}$ %
i\c{c}in azalmayan nitelikteki dizilimler %
%------------------------------(16)-----------------------------------%
\begin{eqnarray}
 1\quad 2 \quad 3 \nonumber\\
 1\quad 2 \quad 2 \nonumber\\
 1\quad 1 \quad 3 \nonumber\\
 1\quad 1 \quad 2 \nonumber\\
 1\quad 1 \quad 1
\end{eqnarray}
%------------------------------(16)-----------------------------------%
bi\c{c}imindedir. Bu yap{\i}ya kar\c{s}\i l\i k gelen katsay{\i} %
dizilimi ise %
%------------------------------(17)-----------------------------------%
\begin{equation}
1\quad 1\quad 2\quad 1\quad 1
\label{eq:j3ise}
\end{equation}
%------------------------------(17)-----------------------------------%
olarak kar\c{s}\i m\i za \c{c}\i kmak\-tad\i r. $j=3$ du\-ru\-mu %
i\c{c}in azalmayan ni\-te\-lik\-te\-ki di\-zi\-lim\-le\-rin sa\-y%
{\i}s{\i} $5$'t\i r. Bu say{\i}, Catalan diziliminde, $n$=$3$ %
durumuna kar\c{s}\i l\i k gelen say\i d\i r. Catalan diziliminin %
genel terimi %
%------------------------------(18)-----------------------------------%
\begin{equation}
 C_{n} = \frac{1}{n+1}{2n \choose n} = \frac{2n!}{(n+1)!n!} \, ,
 \quad n=0,1,2,\ldots
\end{equation}
%------------------------------(18)-----------------------------------%
olarak tan\i mlanmaktad\i r.

$\mathbf{T}_{4}$ i\c{c}in de ayn{\i} ad\i mlar uygulanabilir. Alt%
\"u\c{c}gendeki b\"ut\"un dizilimler %
%------------------------------(19)-----------------------------------%
\begin{eqnarray}
 1\quad 2\quad 3\quad 4\nonumber\\
 1\quad 2\quad 3\quad 3\nonumber\\
 1\quad 2\quad 3\quad 2\nonumber\\
 1\quad 2\quad 3\quad 1\nonumber\\
 1\quad 2\quad 2\quad 4\nonumber\\
 1\quad 2\quad 2\quad 3\nonumber\\
 1\quad 2\quad 2\quad 2\nonumber\\
 1\quad 2\quad 2\quad 1\nonumber\\
 1\quad 2\quad 1\quad 4\nonumber\\
 1\quad 2\quad 1\quad 3\nonumber\\
 1\quad 2\quad 1\quad 2\nonumber\\
 1\quad 2\quad 1\quad 1\nonumber\\
 1\quad 1\quad 3\quad 4\nonumber\\
 1\quad 1\quad 3\quad 3\nonumber\\
 1\quad 1\quad 3\quad 2\nonumber\\
 1\quad 1\quad 3\quad 1\nonumber\\
 1\quad 1\quad 2\quad 4\nonumber\\
 1\quad 1\quad 2\quad 3\nonumber\\
 1\quad 1\quad 2\quad 2\nonumber\\
 1\quad 1\quad 2\quad 1\nonumber\\
 1\quad 1\quad 1\quad 4\nonumber\\
 1\quad 1\quad 1\quad 3\nonumber\\
 1\quad 1\quad 1\quad 2\nonumber\\
 1\quad 1\quad 1\quad 1
\end{eqnarray}
%------------------------------(19)-----------------------------------%
bi\c{c}imdedir. Bu dizilimlerden yaln\i zca ond\"ort adedi azalma%
yan dizilimlerdir. On\-d\"ort say\i s{\i}, be\c{s}inci Catalan sa%
y\i s\i d\i r. Azalmayan dizilimler, %
%------------------------------(20)-----------------------------------%
\begin{eqnarray}
 1\quad 2\quad 3\quad 4\nonumber\\
 1\quad 2\quad 3\quad 3\nonumber\\
 1\quad 2\quad 2\quad 4\nonumber\\
 1\quad 2\quad 2\quad 3\nonumber\\
 1\quad 2\quad 2\quad 2\nonumber\\
 1\quad 1\quad 3\quad 4\nonumber\\
 1\quad 1\quad 3\quad 3\nonumber\\
 1\quad 1\quad 2\quad 4\nonumber\\
 1\quad 1\quad 2\quad 3\nonumber\\
 1\quad 1\quad 2\quad 2\nonumber\\
 1\quad 1\quad 1\quad 4\nonumber\\
 1\quad 1\quad 1\quad 3\nonumber\\
 1\quad 1\quad 1\quad 2\nonumber\\
 1\quad 1\quad 1\quad 1
\end{eqnarray}
%------------------------------(20)-----------------------------------%
bi\c{c}imindedir. Bir azalan dizilimin hangi azalmayan dizilime k%
ar\c{s}\i l\i k geldi\u{g}ini belirlemek i\c{c}in \"ozel ka\-barc%
\i kl{\i} s\i ralay\i \c{s} g\"un\-dem\-de\-dir. Bu i\c{s}lem yap%
\i ld\i ktan sonra, b\"ut\"un a\-zal\-mayan dizilimlere bir tamsa%
y{\i} \c{c}arpan e\c{s}lik ettirilebilir. Bu tamsay{\i} \c{c}arpa%
nlar\i n olu\c{s}turdu\u{g}u dizilim ise % 
%------------------------------(21)-----------------------------------%
\begin{equation}
1\quad 1\quad 2\quad 1\quad 1\quad 3\quad 3\quad 3\quad %
1\quad 1\quad 3\quad 2\quad 1\quad 1 
\label{eq:j4ise}
\end{equation}
%------------------------------(21)-----------------------------------%
bi\c{c}imindedir. Burada g\"ozlem\-len\-me\-si ge\-re\-ken olgu, %
(\ref{eq:j4ise}) diziliminin (\ref{eq:j3ise}) dizilimini ba\c{s}t%
an ba\c{s}\-la\-yan bir alt\-di\-zi\-lim olarak i\c{c}eriyor olma%
s\i d\i r. %

Catalan dizilimi OEIS olarak bilinen an\-siklopedide A000108 olar%
ak kay\i tl\i d\i r. Do\-la\-y{\i}s\i yla, $j.$ Ca\-ta\-lan sa\-y%
\i s{\i}, A000108($j$) olarak g\"osterilebilir. Catalan dizilimin%
in birinci de\u{g}i\c{s}imleri ise A000245 olarak ka\-y{\i}tl{\i}%
d{\i}r. Bu \c{c}al\i \c{s}ma ba\u{g}lam\i nda g\"undeme gelen kat%
say\i lar dizilimi ise A262180 olarak kaydedilmi\c{s}tir. %

$\mathbf{T}_{j}$ i\c{c}in A000108($j$+1) adet azalmayan dizilim o%
lu\c{s}maktad\i r. Bu di\-zi\-lim\-le\-rin tamsay{\i} katsay\i la%
r\i n{\i} i\c{c}eren katsay{\i} dizilimi ise \"o\-zel kabarc\i kl%
{\i} s\i ralay\i \c{s}{\i} kul\-la\-narak belirlenebilir. Bu kats%
ay{\i} dizilimi, $\mathbf{T}_{j-1}$ i\c{c}in olu\c{s}turulan kats%
ay{\i} dizilimini ba\c{s}tan ba\c{s}layan bir altdizilim olarak i%
\c{c}erir. Dolay\i s\i yla katsay{\i} dizilimi A262180, d\"uzensi%
z bir \"u\c{c}gen olu\c{s}turur. Bu d\"uzensiz \"u\c{c}gen %
%------------------------------(22)-----------------------------------%
\begin{eqnarray}
&[1]& 1 \nonumber\\
&[2]& 1 \nonumber\\
&[3]& 2\quad 1\quad 1 \nonumber\\
&[4]& 3\quad 3\quad 3\quad 1\quad 1\quad 3\quad %
2\quad 1\quad 1 \nonumber\\
&\vdots&  \vdots
\end{eqnarray}
%------------------------------(22)-----------------------------------%
yap\i s\i ndad\i r. \"U\c{c}\-ge\-nin $j.$ s{\i}ras{\i}, $j>1$ is%
e, A000245($j$) terimlidir. %  

Herhangi bir azalmayan dizilimin hangi azalabilen dizilim\-lerle %
e\c{s}\-de\-\u{ger} ol\-du\-\u{g}u\-nu bul\-mak i\c{c}in, \"ozel %
kabar\-c\i k\-l{\i} s\i rabozu\c{s} kullan\i labilir. \"Ozel kaba%
rc\i kl{\i} s\i rabozu\c{s}, \"ozel kabarc\i kl{\i} s{\i}\-ra\-la%
y{\i}\c{s}\i n evrik i\c{s}lemidir. Dizili\-min $(j+1)$in\-ci \"o%
\u{g}esinin $j$inci \"o\u{g}esinden en az iki daha b\"uy\"uk oldu%
\u{g}u b\"ut\"un yerler imlenerek s\i rabozu\c{s}a ba\c{s}lanabil%
ir. Bu yer\-ler\-de $(j+1)$'inci \"o\u{g}eden bir \c{c}\i kar\i l%
\i p $j$'inci \"o\u{g}e ile yer de\u{g}i\c{s}\-tirerek yeni bir d%
al yarat\i l\i r. \.I\c{s}lem yeni olu\c{s}an her dizilimde yinel%
enir. E\u{g}er daha \"once elde edilmi\c{s} bir dizilim yeniden e%
lde edilir ise o dalda ilerlenmez. Bu ba\u{g}lamda azalmayan dizi%
lime kar\c{s}\i l\i k gelen b\"ut\"un azalabilen dizilimler bulun%
abilir. %

Buradaki \"onemli a\c{s}amalardan biri, dizilimden ilgili toplamd%
izi katk\i s\i n\i n elde edinimidir. Bu i\c{s}lem, do\u{g}rudan %
dizilimi inceleyerek yap\i labilir, ama bir ara ayg\i t olarak ik%
ili a\u{g}a\c{c}lar\i n kullan\i m{\i} bu i\c{s}lemi kolayla\c{s}%
t\i r\i r. Her azalmayan dizilim i\c{c}in bir ikili a\u{g}a\c{c} %
\"uretimi s\"oz konusudur. Dizilim \"o\u{g}eleri s{\i}\-ra i\-le %
i\c{s}lenir. \.Ilk \"o\u{g}e k\"ok\"u olu\c{s}turur. Bir sonraki %
\"o\u{g}e ayn{\i} de\u{g}erde ise, bu yeni ek\-le\-necek olan \"o%
\u{g}e sol \c{c}ocu\u{g}u olu\c{s}turur. Bir sonraki \"o\-\u{g}en%
in de\u{g}eri, a\u{g}aca konmu\c{s} olan \"o\u{g}enin de\u{g}erin%
den bir b\"uy\"uk ise, sa\u{g} \c{c}ocuk olu\c{s}ur. Ye\-ni gelin%
en \"o\u{g}ede dizilimin bir sonraki terimi incelenir ve ayn{\i} %
ad\i mlar yinelenir. E\u{g}er son eklenen \"o\u{g}e i\-le ye\-ni %
\"o\u{g}e aras\i ndaki de\u{g}i\c{s}im 2 ya da daha b\"u\-y\"uk i%
se, s\i ral{\i} gezini\c{s}te (ing: inorder traversal) ilgi\-li d%
e\u{g}er kadar ilerideki yere konumland\i r\i m ger\c{c}ekle\c{s}%
tirilir. Asl\i nda bu i\c{s}lem, azalabilen dizilimler i\c{c}in d%
e tan\i mlanabilir ve o durumda, azal\i m noktas\i nda s{\i}\-ral%
{\i} gezini\c{s}te daha geride bir konuma konumland\i r\i m ile y%
\"ontem genelle\c{s}tirilebilir. Bu bi\c{c}imde genelle\c{s}ti\-r%
im yap\i l\i nca, \"ozel kabarc\i kl{\i} s{\i}ra\-lay{\i}\c{s} al%
t{\i}nda ayn{\i} dizilimi \"ureten dizilimlerin ay\-n{\i} a\u{g}a%
c{\i} \"urettikleri g\"ozlemlenir. A\u{g}a\c{c} olu\c{s}turuldukt%
an sonra, d\"u\u{g}\"um de\u{g}erlerinin bir \"onemi yoktur. \"On%
emli olan a\u{g}ac\i n yap\i s\i d\i r. D\"u\u{g}\"um de\u{g}erle%
ri a\u{g}ac\i n yap\i s\i nda, \"ozlerini dolayl{\i} olarak bulun%
durur. \.Ikinci, \"u\c{c}\"unc\"u ve d\"ord\"unc\"u d\"or\-d\"ull%
e\c{s}tirilmi\c{s} \i rakg\"or\"ur dizeyleri ba\u{g}\-la\-m{\i}nd%
a g\"undeme gelen a\u{g}a\c{c} yap{\i}lar{\i}, s{\i}\-ra\-s{\i}yl%
a \c{c}izim 1,2 ve 3'te verilmi\c{s}\-tir. $\mathbf{S}_{1}$'e kar%
\c{s}\i l\i k gelen a\u{g}a\c{c} i\-se, bir d\"u\u{g}\"um\-l\"u a%
\u{g}a\c{c}t\i r. %

\begin{figure}
\centering
\caption{$\mathbf{S}_{2}$ ba\u{g}lam\i nda g\"undeme gelen 
ikili a\u{g}a\c{c} yap\i lar{\i}}
\begin{tabular}{|c|c|}
\hline
 (a) 12 & (b) 11 \\
\tikz [baseline=(a.base), tree layout, minimum number of children=2,
sibling distance=5mm, level distance=5mm]
\graph [nodes={circle, inner sep=0pt, minimum size=2mm, fill, as=}]{
a -- {{} ,{b}}
}; &
\tikz [baseline=(a.base), tree layout, minimum number of children=2,
sibling distance=5mm, level distance=5mm]
\graph [nodes={circle, inner sep=0pt, minimum size=2mm, fill, as=}]{
a -- {{b} ,{}}
}; \\
&\\
\hline
\end{tabular}
\end{figure}
%\begin{figure}
%\centering
%\caption{$\mathbf{S}_{2}$ ba\u{g}lam\i nda g\"undeme gelen 
%ikili a\u{g}a\c{c} yap\i lar{\i}}
%\subfigure[12]{
%\tikz [baseline=(a.base), tree layout, minimum number of children=2,
%sibling distance=5mm, level distance=5mm]
%\graph [nodes={circle, inner sep=0pt, minimum size=2mm, fill, as=}]{
%a -- {{} ,{b}}
%};
%}
%\unskip\ \vrule\
%\subfigure[11]{
%\tikz [baseline=(a.base), tree layout, minimum number of children=2,
%sibling distance=5mm, level distance=5mm]
%\graph [nodes={circle, inner sep=0pt, minimum size=2mm, fill, as=}]{
%a -- {{b} ,{}}
%};
%}
%\end{figure}
\begin{figure}
\centering
\caption{$\mathbf{S}_{3}$ ba\u{g}lam\i nda g\"undeme gelen ikili 
a\u{g}a\c{c} yap\i lar{\i}}
\begin{tabular}{|c|c|c|c|c|c|}
\hline
(a) 123 & (b) 122 & (c) 121 & (\c{c}) 113 & (d) 112 & (e) 111 \\
\tikz [baseline=(a.base), tree layout, minimum number of children=2,
sibling distance=5mm, level distance=5mm]
\graph [nodes={circle, inner sep=0pt, minimum size=2mm, fill, as=}]{
{ a -- { , e -- { , f } } }};&
\tikz [baseline=(a.base), tree layout, minimum number of children=2,
sibling distance=5mm, level distance=5mm]
\graph [nodes={circle, inner sep=0pt, minimum size=2mm, fill, as=}]{
{ a -- { , e -- { f} } }};&
\tikz [baseline=(a.base), tree layout, minimum number of children=2,
sibling distance=5mm, level distance=5mm]
\graph [nodes={circle, inner sep=0pt, minimum size=2mm, fill, as=}]{
{ a -- { b , e  } }}; &
\tikz [baseline=(a.base), tree layout, minimum number of children=2,
sibling distance=5mm, level distance=5mm]
\graph [nodes={circle, inner sep=0pt, minimum size=2mm, fill, as=}]{
{ a -- { b , e  } }}; &
\tikz [baseline=(a.base), tree layout, minimum number of children=2,
sibling distance=5mm, level distance=5mm]
\graph [nodes={circle, inner sep=0pt, minimum size=2mm, fill, as=}]{
{ a -- { e --  {, f} } }};&
\tikz [baseline=(a.base), tree layout, minimum number of children=2,
sibling distance=5mm, level distance=5mm]
\graph [nodes={circle, inner sep=0pt, minimum size=2mm, fill, as=}]{
a -- e -- f
};\\
&&&&&\\
\hline
\end{tabular}
\end{figure}
%\begin{figure}
%\centering
%\caption{$\mathbf{S}_{3}$ ba\u{g}lam\i nda g\"undeme gelen ikili 
%a\u{g}a\c{c} yap\i lar{\i}}
%\subfigure[123]{
%\tikz [baseline=(a.base), tree layout, minimum number of children=2,
%sibling distance=5mm, level distance=5mm]
%\graph [nodes={circle, inner sep=0pt, minimum size=2mm, fill, as=}]{
%{ a -- { , e -- { , f } } }};
%}
%\unskip\ \vrule\
%\subfigure[122]{
%\tikz [baseline=(a.base), tree layout, minimum number of children=2,
%sibling distance=5mm, level distance=5mm]
%\graph [nodes={circle, inner sep=0pt, minimum size=2mm, fill, as=}]{
%{ a -- { , e -- { f} } }};
%}
%\unskip\ \vrule\
%\subfigure[121]{
%\tikz [baseline=(a.base), tree layout, minimum number of children=2,
%sibling distance=5mm, level distance=5mm]
%\graph [nodes={circle, inner sep=0pt, minimum size=2mm, fill, as=}]{
%{ a -- { b , e  } }}; 
%}
%\unskip\ \vrule\
%\subfigure[113]{
%\tikz [baseline=(a.base), tree layout, minimum number of children=2,
%sibling distance=5mm, level distance=5mm]
%\graph [nodes={circle, inner sep=0pt, minimum size=2mm, fill, as=}]{
%{ a -- { b , e  } }}; 
%}
%\unskip\ \vrule\
%\subfigure[112]{
%\tikz [baseline=(a.base), tree layout, minimum number of children=2,
%sibling distance=5mm, level distance=5mm]
%\graph [nodes={circle, inner sep=0pt, minimum size=2mm, fill, as=}]{
%{ a -- { e --  {, f} } }};
%}
%\unskip\ \vrule\
%\subfigure[111]{
% \tikz [baseline=(a.base), tree layout, minimum number of children=2,
% sibling distance=5mm, level distance=5mm]
% \graph [nodes={circle, inner sep=0pt, minimum size=2mm, fill, as=}]{
% a -- e -- f
% };
% }
% \end{figure}

\begin{figure}
\centering
\caption{$\mathbf{S}_{4}$ ba\u{g}lam\i nda g\"undeme gelen ikili 
a\u{g}a\c{c} yap\i lar{\i}}
\begin{tabular}{|c|c|c|c|c|c|c|c|}
\hline
(a) 1234 & (b) 1233 & (c) 1232 & (\c{c}) 1231 & (d) 1224 & (e) 1223 & (f) 1222 & (g) 1221 \\
\tikz [baseline=(a.base), tree layout, minimum number of children=2,
sibling distance=5mm, level distance=5mm]
\graph [nodes={circle, inner sep=0pt, minimum size=2mm, fill, as=}]{
{ a -- { , e -- { , f -- { , g} } } }
}; &
\tikz [baseline=(a.base), tree layout, minimum number of children=2,
sibling distance=5mm, level distance=5mm]
\graph [nodes={circle, inner sep=0pt, minimum size=2mm, fill, as=}]{
{ a -- { , e -- { , f -- { g, } } } }
}; &
\tikz [baseline=(a.base), tree layout, minimum number of children=2,
sibling distance=5mm, level distance=5mm]
\graph [nodes={circle, inner sep=0pt, minimum size=2mm, fill, as=}]{
{ a -- { , e -- {f , g  } } }
}; &
\tikz [baseline=(a.base), tree layout, minimum number of children=2,
sibling distance=5mm, level distance=5mm]
\graph [nodes={circle, inner sep=0pt, minimum size=2mm, fill, as=}]{
{ a -- { e, f -- { , g  } } }
}; &
\tikz [baseline=(a.base), tree layout, minimum number of children=2,
sibling distance=5mm, level distance=5mm]
\graph [nodes={circle, inner sep=0pt, minimum size=2mm, fill, as=}]{
{ a -- { , e -- {f , g  } } }
}; &
\tikz [baseline=(a.base), tree layout, minimum number of children=2,
sibling distance=5mm, level distance=5mm]
\graph [nodes={circle, inner sep=0pt, minimum size=2mm, fill, as=}]{
{ a -- { , e -- { f  -- { , g} } } }
}; &
\tikz [baseline=(a.base), tree layout, minimum number of children=2,
sibling distance=5mm, level distance=5mm]
\graph [nodes={circle, inner sep=0pt, minimum size=2mm, fill, as=}]{
{ a -- { , e -- { f  -- {g , } } } }
}; &
\tikz [baseline=(a.base), tree layout, minimum number of children=2,
sibling distance=5mm, level distance=5mm]
\graph [nodes={circle, inner sep=0pt, minimum size=2mm, fill, as=}]{
{ a -- { b , e -- { f  } } }
}; \\
&&&&&&&\\
\hline
(h) 1214 & ({\i}) 1213 & (i) 1212 & (j) 1211 & (k) 1134 & (l) 1133 & (m) 1132 & (n) 1131 \\
\tikz [baseline=(a.base), tree layout, minimum number of children=2,
sibling distance=5mm, level distance=5mm]
\graph [nodes={circle, inner sep=0pt, minimum size=2mm, fill, as=}]{
{ a -- { b , e -- { , f  } } }
};&
\tikz [baseline=(a.base), tree layout, minimum number of children=2,
sibling distance=5mm, level distance=5mm]
\graph [nodes={circle, inner sep=0pt, minimum size=2mm, fill, as=}]{
{ a -- { b , e -- { f  } } }
};&
\tikz [baseline=(a.base), tree layout, minimum number of children=2,
sibling distance=5mm, level distance=5mm]
\graph [nodes={circle, inner sep=0pt, minimum size=2mm, fill, as=}]{
{e -- { f -- {,b}, g}} 
};&
\tikz [baseline=(a.base), tree layout, minimum number of children=2,
sibling distance=5mm, level distance=5mm]
\graph [nodes={circle, inner sep=0pt, minimum size=2mm, fill, as=}]{
{e -- { b -- c, g}} 
};&
\tikz [baseline=(a.base), tree layout, minimum number of children=2,
sibling distance=5mm, level distance=5mm]
\graph [nodes={circle, inner sep=0pt, minimum size=2mm, fill, as=}]{
{ a -- { b , e -- { , f  } } }
};& 
\tikz [baseline=(a.base), tree layout, minimum number of children=2,
sibling distance=5mm, level distance=5mm]
\graph [nodes={circle, inner sep=0pt, minimum size=2mm, fill, as=}]{
{ a -- { b , e -- { f  } } }
};&
\tikz [baseline=(a.base), tree layout, minimum number of children=2,
sibling distance=5mm, level distance=5mm]
\graph [nodes={circle, inner sep=0pt, minimum size=2mm, fill, as=}]{
{e -- { f -- {,b}, g}} 
};&
\tikz [baseline=(a.base), tree layout, minimum number of children=2,
sibling distance=5mm, level distance=5mm]
\graph [nodes={circle, inner sep=0pt, minimum size=2mm, fill, as=}]{
{e -- {  b -- c, g}} 
};\\
&&&&&&&\\
\hline
(o) 1124 & (\"o) 1123 & (p) 1122 & (r) 1121 & (s) 1114 & (\c{s}) 1113 & (t) 1112 & (u) 1111 \\
\tikz [baseline=(a.base), tree layout, minimum number of children=2,
sibling distance=5mm, level distance=5mm]
\graph [nodes={circle, inner sep=0pt, minimum size=2mm, fill, as=}]{
{e -- { f -- {,b}, g}} 
};&
\tikz [baseline=(a.base), tree layout, minimum number of children=2,
sibling distance=5mm, level distance=5mm]
\graph [nodes={circle, inner sep=0pt, minimum size=2mm, fill, as=}]{
b -- { a -- { , f -- { , g  } } } 
};&
\tikz [baseline=(a.base), tree layout, minimum number of children=2,
sibling distance=5mm, level distance=5mm]
\graph [nodes={circle, inner sep=0pt, minimum size=2mm, fill, as=}]{
b -- { a -- { , f -- { g,   } } } 
};&
\tikz [baseline=(a.base), tree layout, minimum number of children=2,
sibling distance=5mm, level distance=5mm]
\graph [nodes={circle, inner sep=0pt, minimum size=2mm, fill, as=}]{
{e -- { f -- {c,b}}} 
};&
\tikz [baseline=(a.base), tree layout, minimum number of children=2,
sibling distance=5mm, level distance=5mm]
\graph [nodes={circle, inner sep=0pt, minimum size=2mm, fill, as=}]{
{e -- { b -- c, g}} 
};& 
\tikz [baseline=(a.base), tree layout, minimum number of children=2,
sibling distance=5mm, level distance=5mm]
\graph [nodes={circle, inner sep=0pt, minimum size=2mm, fill, as=}]{
{e -- { f -- {c,b}}} 
};&
\tikz [baseline=(a.base), tree layout, minimum number of children=2,
sibling distance=5mm, level distance=5mm]
\graph [nodes={circle, inner sep=0pt, minimum size=2mm, fill, as=}]{
{a -- { b -- c -- {,d}}} 
};&
\tikz [baseline=(a.base), tree layout, minimum number of children=2,
sibling distance=5mm, level distance=5mm]
\graph [nodes={circle, inner sep=0pt, minimum size=2mm, fill, as=}]{
{a -- { b -- c -- {d}}} 
};\\
&&&&&&&\\
\hline
\end{tabular}
\end{figure}

% \begin{figure}
% \centering
% \caption{$\mathbf{S}_{4}$ ba\u{g}lam\i nda g\"undeme gelen ikili 
% a\u{g}a\c{c} yap\i lar{\i}}
% \subfigure[1234]{
% \tikz [baseline=(a.base), tree layout, minimum number of children=2,
% sibling distance=5mm, level distance=5mm]
% \graph [nodes={circle, inner sep=0pt, minimum size=2mm, fill, as=}]{
% { a -- { , e -- { , f -- { , g} } } }
% };
% }
% \unskip\ \vrule\
% \subfigure[1233]{
% \tikz [baseline=(a.base), tree layout, minimum number of children=2,
% sibling distance=5mm, level distance=5mm]
% \graph [nodes={circle, inner sep=0pt, minimum size=2mm, fill, as=}]{
% { a -- { , e -- { , f -- { g, } } } }
% };
% }
% \unskip\ \vrule\
% \subfigure[1232]{
% \tikz [baseline=(a.base), tree layout, minimum number of children=2,
% sibling distance=5mm, level distance=5mm]
% \graph [nodes={circle, inner sep=0pt, minimum size=2mm, fill, as=}]{
% { a -- { , e -- {f , g  } } }
% };
% }
% \unskip\ \vrule\
% \subfigure[1231]{
% \tikz [baseline=(a.base), tree layout, minimum number of children=2,
% sibling distance=5mm, level distance=5mm]
% \graph [nodes={circle, inner sep=0pt, minimum size=2mm, fill, as=}]{
% { a -- { e, f -- { , g  } } }
% };
% }
% \unskip\ \vrule\
% \subfigure[1224]{
% \tikz [baseline=(a.base), tree layout, minimum number of children=2,
% sibling distance=5mm, level distance=5mm]
% \graph [nodes={circle, inner sep=0pt, minimum size=2mm, fill, as=}]{
% { a -- { , e -- {f , g  } } }
% };
% }
% \unskip\ \vrule\
% \subfigure[1223]{
% \tikz [baseline=(a.base), tree layout, minimum number of children=2,
% sibling distance=5mm, level distance=5mm]
% \graph [nodes={circle, inner sep=0pt, minimum size=2mm, fill, as=}]{
% { a -- { , e -- { f  -- { , g} } } }
% };
% }
% \unskip\ \vrule\
% \subfigure[1222]{
% \tikz [baseline=(a.base), tree layout, minimum number of children=2,
% sibling distance=5mm, level distance=5mm]
% \graph [nodes={circle, inner sep=0pt, minimum size=2mm, fill, as=}]{
% { a -- { , e -- { f  -- {g , } } } }
% };
% }
% \unskip\ \vrule\
% \subfigure[1221]{
% \tikz [baseline=(a.base), tree layout, minimum number of children=2,
% sibling distance=5mm, level distance=5mm]
% \graph [nodes={circle, inner sep=0pt, minimum size=2mm, fill, as=}]{
% { a -- { b , e -- { f  } } }
% };
% } \\ 
% 
% \subfigure[1214]{
% \tikz [baseline=(a.base), tree layout, minimum number of children=2,
% sibling distance=5mm, level distance=5mm]
% \graph [nodes={circle, inner sep=0pt, minimum size=2mm, fill, as=}]{
% { a -- { b , e -- { , f  } } }
% };
% }
% \unskip\ \vrule\
% \subfigure[1213]{
% \tikz [baseline=(a.base), tree layout, minimum number of children=2,
% sibling distance=5mm, level distance=5mm]
% \graph [nodes={circle, inner sep=0pt, minimum size=2mm, fill, as=}]{
% { a -- { b , e -- { f  } } }
% };
% }
% \unskip\ \vrule\
% \subfigure[1212]{
% \tikz [baseline=(a.base), tree layout, minimum number of children=2,
% sibling distance=5mm, level distance=5mm]
% \graph [nodes={circle, inner sep=0pt, minimum size=2mm, fill, as=}]{
% {e -- { f -- {,b}, g}} 
% };
% }
% \unskip\ \vrule\
% \subfigure[1211]{
% \tikz [baseline=(a.base), tree layout, minimum number of children=2,
% sibling distance=5mm, level distance=5mm]
% \graph [nodes={circle, inner sep=0pt, minimum size=2mm, fill, as=}]{
% {e -- { b -- c, g}} 
% };
% }
% \unskip\ \vrule\
% \subfigure[1134]{
% \tikz [baseline=(a.base), tree layout, minimum number of children=2,
% sibling distance=5mm, level distance=5mm]
% \graph [nodes={circle, inner sep=0pt, minimum size=2mm, fill, as=}]{
% { a -- { b , e -- { , f  } } }
% };
% }
% \unskip\ \vrule\
% \subfigure[1133]{
% \tikz [baseline=(a.base), tree layout, minimum number of children=2,
% sibling distance=5mm, level distance=5mm]
% \graph [nodes={circle, inner sep=0pt, minimum size=2mm, fill, as=}]{
% { a -- { b , e -- { f  } } }
% };
% }
% \unskip\ \vrule\
% \subfigure[1132]{
% \tikz [baseline=(a.base), tree layout, minimum number of children=2,
% sibling distance=5mm, level distance=5mm]
% \graph [nodes={circle, inner sep=0pt, minimum size=2mm, fill, as=}]{
% {e -- { f -- {,b}, g}} 
% };
% }
% \unskip\ \vrule\
% \subfigure[1131]{
% \tikz [baseline=(a.base), tree layout, minimum number of children=2,
% sibling distance=5mm, level distance=5mm]
% \graph [nodes={circle, inner sep=0pt, minimum size=2mm, fill, as=}]{
% {e -- {  b -- c, g}} 
% };
% }
% %\end{figure}
% 
% %\begin{figure}
% %\centering
% %\caption{$\mathbf{S}_{4}$ devam{\i}}
% \subfigure[1124]{
% \tikz [baseline=(a.base), tree layout, minimum number of children=2,
% sibling distance=5mm, level distance=5mm]
% \graph [nodes={circle, inner sep=0pt, minimum size=2mm, fill, as=}]{
% {e -- { f -- {,b}, g}} 
% };
% }
% \unskip\ \vrule\
% \subfigure[1123]{
% \tikz [baseline=(a.base), tree layout, minimum number of children=2,
% sibling distance=5mm, level distance=5mm]
% \graph [nodes={circle, inner sep=0pt, minimum size=2mm, fill, as=}]{
% b -- { a -- { , f -- { , g  } } } 
% };
% }
% \unskip\ \vrule\
% \subfigure[1122]{
% \tikz [baseline=(a.base), tree layout, minimum number of children=2,
% sibling distance=5mm, level distance=5mm]
% \graph [nodes={circle, inner sep=0pt, minimum size=2mm, fill, as=}]{
% b -- { a -- { , f -- { g,   } } } 
% };
% }
% \unskip\ \vrule\
% \subfigure[1121]{
% \tikz [baseline=(a.base), tree layout, minimum number of children=2,
% sibling distance=5mm, level distance=5mm]
% \graph [nodes={circle, inner sep=0pt, minimum size=2mm, fill, as=}]{
% {e -- { f -- {c,b}}} 
% };
% }
% \unskip\ \vrule\
% \subfigure[1114]{
% \tikz [baseline=(a.base), tree layout, minimum number of children=2,
% sibling distance=5mm, level distance=5mm]
% \graph [nodes={circle, inner sep=0pt, minimum size=2mm, fill, as=}]{
% {e -- { b -- c, g}} 
% };
% }
% \unskip\ \vrule\
% \subfigure[1113]{
% \tikz [baseline=(a.base), tree layout, minimum number of children=2,
% sibling distance=5mm, level distance=5mm]
% \graph [nodes={circle, inner sep=0pt, minimum size=2mm, fill, as=}]{
% {e -- { f -- {c,b}}} 
% };
% }
% \unskip\ \vrule\
% \subfigure[1112]{
% \tikz [baseline=(a.base), tree layout, minimum number of children=2,
% sibling distance=5mm, level distance=5mm]
% \graph [nodes={circle, inner sep=0pt, minimum size=2mm, fill, as=}]{
% {a -- { b -- c -- {,d}}} 
% };
% }
% \unskip\ \vrule\
% \subfigure[1111]{
% \tikz [baseline=(a.base), tree layout, minimum number of children=2,
% sibling distance=5mm, level distance=5mm]
% \graph [nodes={circle, inner sep=0pt, minimum size=2mm, fill, as=}]{
% {a -- { b -- c -- {d}}} 
% };
% }
% \end{figure}

A\u{g}a\c{c}tan dizey cebircil yap\i ya ge\-\c{c}i\c{s} i\-\c{c}i%
n s\i ral{\i} gezini\c{s} kullan\i labilir. S\i ral{\i} ge\-zi\-n%
i\c{s}in kural{\i}, her d\"u\u{g}\"um i\-\c{c}in \"on\-ce sol alt%
a\u{g}ac{\i}n i\c{s}lenmesi, sonra d\"u\u{g}\"um\"un kendisinin i%
\c{s}lenmesi ve daha sonra sa\u{g} alta\u{g}ac\i n i\c{s}lenmesid%
ir. Bu olgu, ba\c{s}lang\i \c{c} noktas{\i} olarak, k\"okten olab%
ildi\u{g}ince kez sol \c{c}ocu\u{g}a ge\c{c}i\c{s}e neden olur. B%
ir d\"u\u{g}\"um, yaln{\i}z ba\c{s}\i na, bir d\"ord\"ulle\c{s}ti%
rime kar\c{s}\i l\i k gelmektedir. Daha somut s\"oylemek gerekirs%
e, bir d\"u\u{g}\"um, yaln{\i}z ba\c{s}{\i}na d\"u\-\c{s}\"un\"ul%
d\"u\u{g}\"unde, $ \lfloor \mathbf{F} , \mathbf{a} \rceil $ ya\-p%
\i s\i n{\i} betimler. Dolay\i s{\i}yla, s\i ral{\i} gezini\c{s}i%
n ba\c{s}lang\i \c{c} noktas{\i} budur. Soldan ebe\-veyne ge\c{c}%
i\c{s}, sol alta\u{g}a\c{c}taki yap\i n\i n d\"ord\"ulle\c{s}tiri%
mini olu\c{s}turur. S{\i}ra\-l{\i} gezini\c{s} ba\u{g}lam\i nda, %
d\"u\u{g}\"umden sa\u{g} \c{c}o\-cu\u{g}a ge\-\c{c}i\c{s} i\-se s%
a\u{g} alta\u{g}ac\i n olu\c{s}turdu\u{g}u yap{\i} i\-le sa\u{g}d%
an \c{c}arpmaya kar\c{s}\i l\i k gelir. Bu ba\u{g}lamda, a\u{g}ac%
a yaln\i zca bakarak bile, nas\i l bir dizey cebircil yap{\i} olu%
\c{s}turdu\u{g}u ile ilgili bir\c{s}eyler s\"oy\-lemek o\-la\-nak%
l\i d\i r. %

A\u{g}a\c{c}\-tan di\-zey ceb\-rine ge\-\c{c}i\c{s} i\-\c{c}in, a%
\u{g}ac\i n dizin g\"osterilimi kullan\i labilir. A\u{g}ac\i n di%
zin g\"osterilimi, b\"ut\"un a\u{g}ac{\i} \"u\c{c} \"o\-\u{g}eli %
bir dizin olarak ele al\i r. Bu dizinin birinci \"o\u{g}esi sol a%
lta\u{g}a\c{c}a kar\c{s}\i l\i k gelen dizin, ikinci \"o\u{g}esi %
k\"ok, \"u\c{c}\"unc\"u \"o\u{g}esi ise sa\u{g} alta\u{g}a\c{c}a %
kar\c{s}\i l\i k gelen dizindir. Alta\u{g}a\c{c}lar i\c{c}in de a%
yn{\i} i\c{s}lem uygulan\i r ve i\c{c} i\c{c}e dizin\-ler yap{\i}%
s{\i} kurulur. Buradan dizey cebircil yap{\i}\-ya ge\c{c}i\c{s} i%
\c{c}in, di\-zin i\c{s}\-le\-me\-ye da\-ha uy\-gun o\-lan i\c{s}l%
ev\-cil buy\-ruk\-di\-zi\-le\-yi\c{s} yak\-la\-\c{s}{\i}\-m{\i} k%
ul\-la\-n{\i}\-la\-bi\-le\-ce\-\u{g}i gi\-bi, bir k{\i}\-sa\-yol %
o\-la\-rak, simge katar{\i} i\c{s}lenimi y\"ontemi kullan{\i}labi%
lir. Dizin g\"osterilimi bir simge katar{\i} olarak d\"u\c{s}\"un%
\"ul\"up, \"uzerinde baz{\i} yal\i n i\c{s}lemler yap{\i}l\i r is%
e, ilgili i\c{c} i\-\c{c}e i\c{s}\-lev \c{c}a\u{g}r\i lar{\i} yap%
\i s{\i} olu\c{s}turulabilir. O\-lu\c{s}\-tu\-rulan bu k\i sa bet%
ik \c{c}al\i \c{s}t\i r\i larak, dizeycil katk{\i} elde edilebili%
r. \.Ilgili betik yap{\i}s{\i}n{\i}n bu ba\u{g}\-lamda kullan{\i}%
m{\i} a\c{s}a\u{g}\i daki \"ornek betikte g\"osterilmi\c{s}tir. %
\begin{verbatim}
>> package("Squtelmat");
>> alias(d = combinat::trees::dummyLabel):
>> bt:=combinat::binaryTrees([d, [d, [d, [d, [], []], []],[d, [], []]], []]);

                                                                        o
                                                                       /
                                                                      /\
                                                                     /
>> Squtelmat::tree2nestedCalls(bt);

                  "(Squtelmat::sq(F,Squtelmat::sq(F,Squtelmat::sq(F,Squtelmat::sq(F,
                  a) * a) * a) * Squtelmat::sq(F,a) * a))"
\end{verbatim}

\section{Betikleyi\c{s}}

Be\-tik\-le\-yi\c{s}, \textit{U\-bun\-tu Li\-nux} or\-tam\i nda, %
\textit{MuPAD Pro Version 4.0.6} ile ya\-p\i l\-m{\i}\c{s}\-t{\i}%
r. \textit{PET}, \textit{Squtelmat} ve \textit{BelowDiagonalTrave%
rsal} adlar\i nda \"u\c{c} ayr{\i} okunakl\i k ve bu okunakl\i kl%
ar{\i} kullanan betik olu\c{s}turulmu\c{s}tur. S\i nay\i\c{s} ama%
\c{c}l{\i} olarak da ad\i mlar at\i lm\i \c{s}t\i r. Bunlardan bi%
ri, OEK ba\u{g}lam\i nda elde edilen denklem \c{c}\"oz\"um\"un\"u%
n ayr\i kla\c{s}t\i r\i m tabanl{\i} y\"ontemlerle kar\-\c{s}\i l%
a\c{s}t\i r\i m\i d\i r. Bu ba\u{g}lamdaki \c{c}al\i \c{s}malar d%
erinle\c{s}tirilecektir. Bir di\u{g}er s\i nay\i \c{s} ise, d\"or%
d\"ulle\c{s}tirim \"uzerinedir. D\"ord\"ulle\c{s}tirim kullan\i l%
an OEK y\"ontemi ile kullan\i lmayan OEK y\"ontemi kar\-\c{s}\i l%
a\c{s}t\i r\i lm\i \c{s}, sonu\c{c}lar aras{\i}n\-da tam \"ort\"u%
\c{s}\"um g\"or\"ulm\"u\c{s}t\"ur. Bu s\i nay\i \c{s}larda $j=8$ %
durumuna kadar \c{c}\i k\i lm\i \c{s}t\i r. %

\section{\.Ileri konular}

Y\"ontemi kullanarak $j=8$ ya da $j=9$ durumundan \"oteye ge\c{c}%
mek, yap\i n\i n \c{s}u anki durumuyla olanakl{\i} g\"or\"unmemek%
tedir. Bunu olanakl{\i} k\i lmak i\c{c}in kuramc\i l geli\c{s}tir%
imler gereklidir. Betikte en \"onemli zaman al\i c{\i} olgu, kats%
ay{\i} diziliminin belirleni\c{s}idir. Bu katsay{\i} diziliminin %
std tak{\i}m\i ndan ba\u{g}\i ms\i z oldu\u{g}u u\-nutulmamal\i d%
\i r. Bu nedenle, bir kez belir\-lenimi ve tutamak ya da veritaba%
n\i nda saklan\i m{\i} yeterlidir. Di\u{g}er \"o\-nem\-li zaman a%
l\i c{\i} olgu ise dizey ya da y\"oney katk\i lar\i n belirle\-ni%
mi ve toplan\i m\i d\i r. Burada baz{\i} k\"u\c{c}\"uk iyile\c{s}%
tirimler yap\i labilir. Ama as\i l h\i zland\i r\i m, ancak d\"or%
d\"ulle\c{s}tirilmi\c{s} \i rakg\"or\"ur dizeyleri a\-ras\i nda b%
ir \"ozyineleyi\c{s} elde edinimi ile olabilir. Bu \"oz\-yineleyi%
\c{s}in elde edinimi i\c{c}in ad\i mlar at\i lm\i \c{s}t\i r. %

Bunun i\c{c}in ba\-z{\i} g\"oz\-lem\-ler\-le yola \c{c}\i k\i lma%
l\i d{\i}r. \.Ilk g\"oz\-lem $\mathbf{S}_{j+1}$ i\-\c{c}e\-ri\-si%
nde $\mathbf{S}_{j}$'nin b\"u\-t\"u\-n\"u\-n\"un d\"or\-d\"ul\-le%
\c{s}\-ti\-ri\-mi\-nin bulun\-du\u{g}u\-dur. Bu\-nun an\-lam{\i} %
$\mathbf{S}_{j}$'nin b\"u\-t\"un terimlerinin d\"ord\"ulle\c{s}ti%
riminin ayn{\i} katsay{\i}larla ken\-di\-ni g\"os\-ter\-me\-si\-d%
ir. \.I\-kin\-ci ol\-gu ise $\mathbf{S}_{j}$'nin b\"u\-t\"un teri%
m\-le\-ri\-nin sol\-dan $\lfloor \mathbf{F} , \mathbf{a} \rceil$ %
ile \c{c}ar\-p{\i}m{\i}n{\i}n ay\-n{\i} kat\-sa\-y{\i}\-lar\-la, %
$\mathbf{S}_{j+1}$ i\c{c}inde bulunma\-s{\i}d{\i}r. Ge\-ri kalan %
(A000108($j$+2)$-2\times$A000108($j$+1)) te\-rim ise yine somut olarak %
$\mathbf{S}_{j}$ ile ili\c{s}kilendi\-ri\-le\-bil\-me\-si\-ne kar%
\c{s}\i n daha \c{c}apra\c{s}\i k yap\i dad\i r. %

Bir yaz\i m olgusu olarak %
%------------------------------(23)-----------------------------------%
\begin{equation}
 \mathbf{S}_{j} \equiv \left(\mathbf{F},\mathbf{a}\right)^{dj}
\end{equation}
%------------------------------(23)-----------------------------------%
g\"osterilimi kullan\i lacakt\i r. \"Usteldeki $d$ simgesi d\"ord%
\"ulle\c{s}tirimi \c{c}a\u{g}r\i \c{}st\i rmakta, bu simgenin yan%
\i ndaki say\i l ise ka\c{c}\i nc{\i} d\"ord\"ulle\c{s}ti\-ril\-m%
i\c{s} \i rakg\"or\"ur dizeyi oldu\u{g}unu vurgulamaktad\i r. \.I%
kinci d\"ord\"ulle\c{s}tirilmi\c{s} \i rakg\"or\"ur dizeyi, birin%
cisi bi\c{c}iminden %
%------------------------------(24)-----------------------------------%
\begin{equation}
 \left(\mathbf{F},\mathbf{a}\right)^{d2} = 
 \lfloor \mathbf{F} , \left(\mathbf{F},
 \mathbf{a}\right)^{d1}\mathbf{a} \rceil
 + \lfloor \mathbf{F} , 
 \mathbf{a} \rceil \left(\mathbf{F},\mathbf{a}\right)^{d1}
\end{equation}
%------------------------------(24)-----------------------------------%
olarak yaz\i labilir. Elbette %
$\left(\mathbf{F},\mathbf{a}\right)^{dj}$ dizeyleri aras\i ndaki %
olas{\i} ili\c{s}kilendirimler e\c{s}siz de\u{g}ildir. Burada \"o%
nerilen, olas{\i} ili\c{s}kilendirim bi\c{c}imlerinden yaln{\i}zc%
a biridir. $\left(\mathbf{F},\mathbf{a}\right)^{d3}$ i\-\c{c}in i%
se gerek $\left(\mathbf{F},\mathbf{a}\right)^{d2}$ ve gerek %
$\left(\mathbf{F},\mathbf{a}\right)^{d1}$ dizeylerini i\c{c}inde %
bar\i nd\i ran %
%------------------------------(25)-----------------------------------%
\begin{equation}
 \left(\mathbf{F},\mathbf{a}\right)^{d3} =
 \lfloor \mathbf{F} , \mathbf{a} \rceil 
 \left(\mathbf{F},\mathbf{a}\right)^{d2}
 + 2 \lfloor \mathbf{F} , \left(\mathbf{F},%
 \mathbf{a}\right)^{d1}\mathbf{a} \rceil
 \left(\mathbf{F},\mathbf{a}\right)^{d1}
 + \lfloor \mathbf{F} , \left(\mathbf{F},%
 \mathbf{a}\right)^{d2}\mathbf{a} \rceil
\end{equation}
%------------------------------(25)-----------------------------------%
ili\c{s}kilendirimi ortaya konulabilir. Bir sonraki dizey ise %
%------------------------------(26)-----------------------------------%
\begin{equation}
 \left(\mathbf{F},\mathbf{a}\right)^{d4} = 
 \lfloor \mathbf{F} , \mathbf{a} \rceil 
 \left(\mathbf{F},\mathbf{a}\right)^{d3}
 + 3 \lfloor \mathbf{F} , \left(\mathbf{F},%
 \mathbf{a}\right)^{d2}\mathbf{a} \rceil
 \lfloor \mathbf{F} , \mathbf{a} \rceil 
 + 3 \lfloor \mathbf{F} , \lfloor \mathbf{F} , 
 \mathbf{a} \rceil \mathbf{a} \rceil
 \left(\mathbf{F},\mathbf{a}\right)^{d2}
 + \lfloor \mathbf{F} , \left(\mathbf{F},%
 \mathbf{a}\right)^{d3}\mathbf{a} \rceil
\end{equation}
%------------------------------(26)-----------------------------------%
yap\i s\i ndad\i r. Bu bi\c{c}imde ilerlenebilir. Bir sonraki diz%
ey ise %
%------------------------------(27)-----------------------------------%
\begin{eqnarray}
 \left(\mathbf{F},\mathbf{a}\right)^{d5} &=&
 \lfloor \mathbf{F} , \mathbf{a} \rceil 
 \left(\mathbf{F},\mathbf{a}\right)^{d4}
 + 4 \lfloor \mathbf{F} , \left(\mathbf{F},%
 \mathbf{a}\right)^{d3}\mathbf{a} \rceil
 \lfloor \mathbf{F} , \mathbf{a} \rceil
 + 6 \lfloor \mathbf{F} , \left(\mathbf{F},%
 \mathbf{a}\right)^{d2}\mathbf{a} \rceil
 \left(\mathbf{F},\mathbf{a}\right)^{d2} \nonumber\\
 &+& 4  \lfloor \mathbf{F} ,  \lfloor \mathbf{F} , 
 \mathbf{a} \rceil  
 \mathbf{a} \rceil \left(\mathbf{F},\mathbf{a}\right)^{d3}
 +  \lfloor \mathbf{F} , %
 \left(\mathbf{F},\mathbf{a}\right)^{d4}\mathbf{a} \rceil
\end{eqnarray}
%------------------------------(27)-----------------------------------%
yap\i s\i nda g\"undeme gelir. Bir sonraki di\-zeyin yap{\i}s{\i}%
n{\i}n g\"ozle ve ka\u{g}\i t kalemle be\-lirleni\c{s}i, terim sa%
y\i s\i n\i n \c{c}oklu\u{g}undan dolay{\i} zor\-dur. Ya\-p\i lma%
s{\i} gereken, buraya kadar elde edilen ve do\u{g}rulu\u{g}u g\"o%
zle denetlenebilen yap\i n\i n nas\i l bir \"or\"unt\"u or\-taya %
\c{c}\i kard\i \u{g}\i n{\i} g\"undeme getirmektir. Bu \"o\-r\"un%
t\"u kullan\i larak di\u{g}er dizeyler de elde edilebilir. Katsay%
\i lar olarak, ikiterimli a\c{c}\i l\i m{\i} kat\-say\i lar{\i} g%
\"undemdedir. Terimler ise, daha k\"u\c{c}\"uk s\i rasay\i l{\i} %
$\left(\mathbf{F},\mathbf{a}\right)^{dj}$ dizey\-leri\-ne ol\-duk%
\c{c}a somut kurallarla ba\u{g}l\i d\i r. Son\-ra\-ki iki dizey i%
\c{c}in, \"or\"unt\"u temel al\i narak, \"o\-ne\-rilen yap\i lar %
a\c{s}a\u{g}\i daki gibidir. %
%------------------------------(28)-----------------------------------%
\begin{eqnarray}
 \left(\mathbf{F},\mathbf{a}\right)^{d6} &=&
 \lfloor \mathbf{F} , \mathbf{a} \rceil
 \left(\mathbf{F},\mathbf{a}\right)^{d5}
 + 5 \lfloor \mathbf{F} , \left(\mathbf{F},%
\mathbf{a}\right)^{d4}\mathbf{a} \rceil
 \lfloor \mathbf{F} , \mathbf{a} \rceil 
 + 5 \lfloor \mathbf{F} , \lfloor \mathbf{F} , 
 \mathbf{a} \rceil \mathbf{a} \rceil
 \left(\mathbf{F},\mathbf{a}\right)^{d4}
 + 10  \lfloor \mathbf{F} ,  \lfloor \mathbf{F} , 
 \lfloor \mathbf{F} 
 , \mathbf{a} \rceil \mathbf{a} \rceil \mathbf{a} \rceil 
 \left(\mathbf{F},\mathbf{a}\right)^{d3} \nonumber\\
 &+& 10  \lfloor \mathbf{F} ,  \lfloor \mathbf{F} , 
 \left(\mathbf{F},\mathbf{a}\right)^{d3}\mathbf{a} 
 \rceil\mathbf{a} \rceil 
 \lfloor \mathbf{F} , \mathbf{a} \rceil
 +  \lfloor \mathbf{F} ,  
 \left(\mathbf{F},\mathbf{a}\right)^{d5}\mathbf{a} \rceil
\label{eq:dd6}
\end{eqnarray}
%------------------------------(28)-----------------------------------%
%------------------------------(29)-----------------------------------%
\begin{eqnarray}
 \left(\mathbf{F},\mathbf{a}\right)^{d7} &=&
 \lfloor \mathbf{F} , \mathbf{a} \rceil
 \left(\mathbf{F},\mathbf{a}\right)^{d6}
 + 6 \lfloor \mathbf{F} , 
 \left(\mathbf{F},\mathbf{a}\right)^{d5}\mathbf{a} \rceil
 \lfloor \mathbf{F} , \mathbf{a} \rceil 
 + 6 \lfloor \mathbf{F} , 
 \lfloor \mathbf{F} , \mathbf{a} \rceil \mathbf{a} \rceil
 \left(\mathbf{F},\mathbf{a}\right)^{d5}
 + 15  \lfloor \mathbf{F} ,  
 \lfloor \mathbf{F} , \lfloor \mathbf{F} 
 , \mathbf{a} \rceil \mathbf{a} \rceil \mathbf{a} \rceil 
 \left(\mathbf{F},\mathbf{a}\right)^{d4} \nonumber\\
 &+& 15  \lfloor \mathbf{F} ,  
 \lfloor \mathbf{F} , 
 \left(\mathbf{F},\mathbf{a}\right)^{d4}\mathbf{a} 
 \rceil\mathbf{a} \rceil 
 \lfloor \mathbf{F} , \mathbf{a} \rceil
 + 20 \lfloor \mathbf{F} , 
 \left(\mathbf{F},\mathbf{a}\right)^{d3}\mathbf{a} \rceil
 \left(\mathbf{F},\mathbf{a}\right)^{d3}
 +  \lfloor \mathbf{F} , 
 \left(\mathbf{F},\mathbf{a}\right)^{d6}\mathbf{a} \rceil \, .
 \label{eq:dd7}
\end{eqnarray}
%------------------------------(29)-----------------------------------%
(\ref{eq:dd6}) ve (\ref{eq:dd7}) ba\u{g}\i nt\i lar{\i} ile ilgil%
i do\u{g}rulay\i \c{s} \c{c}al\i \c{s}mas{\i} yap\i lacakt\i r. A%
ma kolay s\i nay\i \c{s}lar hemen yap\i labilir. \"Or\-ne\u{g}in %
$\left(\mathbf{F},\mathbf{a}\right)^{d6}$ dizeyinin %
%------------------------------(30)-----------------------------------%
\begin{equation}
10  \lfloor \mathbf{F} ,  \lfloor \mathbf{F} , \lfloor \mathbf{F} 
 , \mathbf{a} \rceil \mathbf{a} \rceil \mathbf{a} \rceil 
 \left(\mathbf{F},\mathbf{a}\right)^{d3}                                                                     
\end{equation}
%------------------------------(30)-----------------------------------%
terimini inceleme alt\i na alal\i m. Bu terimin yapt\i \u{g}{\i} %
i\c{s}lem, \"u\c{c}\"unc\"u d\"ord\"ulle\c{s}tirilmi\c{s} \i rakg%
\"or\"ur dizeyini soldan \"u\c{c}l\"u d\"ord\"ulle\c{s}tirim ile %
\c{c}arpmakt\i r. Bu ba\u{g}lamda %
%------------------------------(31)-----------------------------------%
\begin{equation}
  1\quad 2\quad 3 \rightarrow 1\quad 1\quad 1\quad 4\quad 5\quad 6
\end{equation}
%------------------------------(31)-----------------------------------%
olmaktad\i r. Soldan \"u\c{c}l\"u d\"ord\"ulle\c{s}\-ti\-rim ile %
\c{c}arp\i m i\c{s}lemi bu bi\c{c}imde tan\i mland\i \u{g}\i na g%
\"ore katsay{\i} da tutarl{\i} olmal\i d\i r. \"O\-zel kabarc\i k%
l{\i} s\i rabozu\c{s} ile bu dizilimin e\c{s}de\u{g}erlilik k\"um%
esinde 10 adet dizilim oldu\u{g}u g\"osterilebilir. Bu di\-ziliml%
er %
%------------------------------(32)-----------------------------------%
\begin{eqnarray}
 S_{1,1,1,4,5,6} &=& \left\{[1,1,1,4,5,6],[1,1,3,1,5,6],%
[1,2,1,1,5,6],[1,1,3,4,1,6],[1,1,3,4,5,1],%
[1,2,1,4,1,6],\right.\nonumber\\
 &&\left.[1,2,1,4,5,1],[1,2,3,1,1,6],%
[1,2,3,1,5,1],[1,2,3,4,1,1]\right\}
\end{eqnarray}
%------------------------------(32)-----------------------------------%
dizilimleridir. Dolay\i s\i yla, bu kat\-k{\i} i\-\c{c}in tu\-tar%
l{\i}l\i k s\"oz konusudur. %

E\u{g}er 
%------------------------------(33)-----------------------------------%
\begin{equation}
 \left(\mathbf{F},\mathbf{a}\right)^{d0} = \mathbf{I}
\end{equation}
%------------------------------(33)-----------------------------------%
oldu\u{g}u g\"oz \"on\"unde bulundurulursa, 
%------------------------------(34)-----------------------------------%
\begin{equation}
 \left(\mathbf{F},\mathbf{a}\right)^{dj} = \sum_{k=0}^{j-1} 
 {j-1 \choose k}  \lfloor \mathbf{F} , 
 \left(\mathbf{F},\mathbf{a}\right)^{dk}\mathbf{a} \rceil
 \left(\mathbf{F},\mathbf{a}\right)^{d(j-1-k)} \, ,  
\quad j=1,2,\ldots
\label{eq:yoo}
\end{equation}
%------------------------------(34)-----------------------------------%
oldu\u{g}u g\"or\"ulebilir. (\ref{eq:yoo}) ba\u{g}\i nt\i s{\i},
bir dizeycil \"ozyineleyi\c{s}tir.
Bu \"ozyineleyi\c{s} do\u{g}rusal
de\u{g}ildir ve yerel de\u{g}ildir. 

\section{Bilimsel yaz\i n}

Olas\i l\i ksal evrim kuram{\i} (OEK) ile ilgili \c{c}al\i \c{s}m%
alar\i n say{\i}s{\i} artmaktad\i r. \"Ozellikle, nicem dizgeleri%
n beklenen de\u{g}er devinimi ba\u{g}lam\i nda \c{c}\"oz\"um\"u %
\"uzerine g\"uncel \c{c}al\i \c{s}malar bulunmaktad\i r. %

Burada, yaln\i zca bu \c{c}al\i \c{s}ma ile birincil d\"uzeyde il%
gili kaynaklara k\i saca de\u{g}inilecektir. \.Ilgile\-nen okuyuc%
u, yaz\i larda kullan\i lan kaynaklar{\i} inceleyebilir ya da ara%
ma ara\c{c}lar{\i} kullanarak, belirli alanlarda derinle\c{s}me y%
oluna gidebilir. %

Sa\u{g} yan{\i} ikinci derece \c{c}ok\c{c}okterimli olan belir\-t%
ik s\i radan t\"urevli denklem tak\i mlar{\i}n\i n \c{c}\"o\-z\"u%
m\"u ile ilgili bilgi edinmek i\c{c}in Co\c{s}ar G\"o\-z\"uk\i rm%
{\i}z{\i} ve Metin Demiralp'in \textit{Journal of Mathematical Ch%
emistry}'de yay\i nlanan yaz\i lar{\i} incelenmelidir %
\cite{p1,p2}. D\"ord\"ul%
le\c{s}\-ti\-rim ile ilgili ise bildiri d\"uzeyinde \"onemli kayn%
aklar eri\c{s}ilebilir durum\-da\-d\i r \cite{MD2014Dec,cg2015, 
mek2015}. %
D\"ord\"ulle\c{s}tirim %
ba\u{g}lam\i nda g\"undeme gelen katsay\i lar\i n o\-lu\c{s}turdu%
\u{g}u dizilim, Co\c{s}ar G\"oz\"uk\i rm\i z{\i} ta\-ra\-f\i n\-d%
an \textit{Online Encyclopedia of Integer Sequences}'a 
ek\-len\-mi\c{s}tir \cite{encent, enc}. %
Co\c{s}ar G\"oz\"uk\i rm\i z{\i} ta\-raf\i n\-dan o\-l%
u\c{s}turulan, sa\u{g} yan{\i} ikinci derece \c{c}ok\c{c}okteriml%
i olan belirtik s\i ra\-dan t\"urevli denklem tak\i mlar\i n\i n % 
ba\c{s}lang\i \c{c} de\-\u{g}er so\-rununu yakla\c{s}\i k olarak %
\c{c}\"ozen betik ise \textit{GitHub}'da eri\c{s}ilebilir durumda%
d\i r \cite{code}. %

\"On\"um\"uzdeki y\i llarda OEK ve \c{c}ev\-re\-sindeki konular b%
a\u{g}lam\i nda yap\i lan \c{c}al\i \c{s}malar\i n artmas{\i}; ya%
z{\i} ve bilimsel toplant\i lar ba\u{g}lam\i nda olu\c{s}an 
etkile\c{s}imle%
r ile kuram\i n kullan\i m\i n\i n yayg\i nla\c{s}mas{\i} beklenm%
ektedir. %

\bibliographystyle{unsrturl}  

%%%%%%%%%%%%%%%%%%%%%%%%%%%%%%%%%%%%%%%%%%%
%% You probably want to use your own bibtex database here
%%%%%%%%%%%%%%%%%%%%%%%%%%%%%%%%%%%%%%%%%%%
\bibliography{anlatim}

\end{document}